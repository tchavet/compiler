\documentclass[a4paper,11pt]{article}
\usepackage[english]{babel}
\usepackage{natbib}
\usepackage{url}
\usepackage[utf8x]{inputenc}
\usepackage{amsmath}
\usepackage{graphicx}
\graphicspath{{img/}}
\usepackage{parskip}
\usepackage{subfig}
\usepackage{float}
\usepackage{tabularx}
\usepackage{fancyhdr}
\usepackage{parskip}
\usepackage{indentfirst}
\usepackage{color}
\usepackage[dvipsnames]{xcolor}
\setlength{\parindent}{15pt}
\usepackage{vmargin}
\usepackage{framed}
%\setmarginsrb{3 cm}{2.5 cm}{3 cm}{2.5 cm}{1 cm}{1.5 cm}{1 cm}{1.5 cm}%

\usepackage{hyperref}

\title{INFO085 - Compiler \\ Semantic Analysis }


\date{\today}

\author{Thibaut \textsc{Chavet} \and Maxim \textsc{Henry} \and Maxime \textsc{Massart}}



\begin{document}

\begin{figure}
\begin{center}
\includegraphics[scale = 1.2]{logoulg}
\end{center}
\end{figure}

\maketitle
\thispagestyle{empty}
\setcounter{page}{0}

% %%%%%%%%%%%%%%%%%%%%%%%%%%%%%%%%%%%%%%%%%%%%%%%%%%%%%%%%%%%%%%%%%%%%%%%%%%%%%%%%%%%%%%%%%
% 
% \begin{abstract} 
% 
% \end{abstract}
% 
% 
% %%%%%%%%%%%%%%%%%%%%%%%%%%%%%%%%%%%%%%%%%%%%%%%%%%%%%%%%%%%%%%%%%%%%%%%%%%%%%%%%%%%%%%%%%
% 
% \tableofcontents
% \pagebreak
% 
% %%%%%%%%%%%%%%%%%%%%%%%%%%%%%%%%%%%%%%%%%%%%%%%%%%%%%%%%%%%%%%%%%%%%%%%%%%%%%%%%%%%%%%%%%

\section{Introduction}
This project is done for the compiler course given by Prof. Geurts during the academic year 2016-2017. 
We are asked to implement a vsop compiler. The vsop is defined on the course website.

\section{Our compiler}
We made improvments during the holidays to keep on improving on good basis. We will present the improvments
by the previous impacted parts already done for this project.

\subsection{Lexical Analysis}
We had an issue with this lexical part. As the input of this part is not correct in the vsop formalism,
our C++ bison yyparse() calls itself yylex. We could have a first parsing phase that would be done twice
but for performances, it is not a good option. If you ask for it, we can do it but it is not optimal.

\subsection{Syntax Analysis}
We checked this part because we tried not to break the previous step so we were not up to date. 
Now we succeeded in all automated tests. We also refactored the classes and expressions to get Nodes in the nodes/ directory.

\subsection{Semantic parsing}
For this part, we implemented this step in three phases. First we implemented the Classes checking, 
then the types checking while retrieving them. After that, we determined the non-redefinition of methods or variables.

\subsubsection{Classes checking}
In this first part, we used a 2D array that contains the Class ID (string) and the ClassNode 
associated to it. 
It is useful to know if this type has been declared in the following part. It also allows to 
know if all classes extending others are all declared.
  
\subsubsection{Types retrieving and checking}
We needed mechanisms to know the closest common parent that we use to determine the output 
of a dynamic instanciation. 
This step has the purpose of checking that all types are correctly implemented.

\subsubsection{Redefinition}
We have to determine if the  fields are not redefined if they are not of the same type. 
The methods can be overwritten if they agree to certain conditions. We implemented by going through the nodes
and asking the parent classes (if needed) for knowing if it is a redefinition.
We also check the arguments of the methods during this step.

\section{Conclusion}
We had fun doing this project even though we found very difficult to collaborate as the two steps were
to be done and as it depends on the previous test that has also been improved. We also changed our implementation
structure quite a lot giving the fact that we could improve the given solution.


\end{document}